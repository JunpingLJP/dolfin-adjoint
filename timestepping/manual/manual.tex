\documentclass[a4paper]{book}
\usepackage[left=20mm,right=20mm]{geometry}

\usepackage{amsmath}
\usepackage{amssymb}
\usepackage{array}
\usepackage{graphicx}
\usepackage{hyperref}
\usepackage{listings}
\usepackage{longtable}
\usepackage{natbib}
\usepackage{nicefrac}
\usepackage[svgnames]{xcolor}

\hypersetup{
  colorlinks = true,
  citecolor = DarkBlue,
  linkcolor = DarkBlue,
  urlcolor = DarkBlue
}

\lstset{
  language = Python,
  basicstyle = \ttfamily\footnotesize,
  frame = single,
  commentstyle = \color{Grey},
  keywordstyle = \color{blue},
  stringstyle = \color{DarkViolet},
  keepspaces = true,
  showspaces = false,
  showstringspaces = false
}

\newcommand{\version}{1.6.0+}

\begin{document}

\begin{titlepage}
\begin{center}

\Huge{\verb+timestepping+ Manual} \\[0.02\textheight]
\large{The high-level representation of transient finite element models with DOLFIN} \\[0.02\textheight]
\Huge{Version \version} \\[0.15\textheight]
\vfill
\large{\today}

\end{center}
\end{titlepage}

\tableofcontents

\chapter{Introduction}

\section{Overview}

The \verb+timestepping+ Python library is an extension for DOLFIN, adding a
high-level representation for time dependent finite element models.
Specifically, the \verb+timestepping+ library enables the rapid development of
efficient timestepping finite element models using a symbolic representation of
the entire model discretisation. The FEniCS system (for which DOLFIN acts as a
front-end) uses automated code generation to convert a symbolic representation
of a finite element spatial discretisation into a working model. The
\verb+timestepping+ library builds on this principle, adding a syntax to enable
a symbolic representation of a temporal discretisation. For further details
see \citet{maddison2014}.

A machine interpretable representation of a discretisation is amenable to
automated analysis and manipulation. In particular, if a symbolic representation
of a model time discretisation is available then a number of optimisation
strategies can be applied automatically. This enables time independent data,
which may be expensive to recompute, to be computed once and cached ahead of the
execution of the main model time loop. Moreover, the FEniCS system supplies
symbolic manipulation tools which enable one to differentiate and adjoin
spatially discretised equations. This functionality can be combined with a
high-level representation of the temporal discretisation to enable the automatic
derivation of an associated discrete adjoint model. Since the resulting adjoint
model itself has a symbolic representation, it is also amenable to the
application of optimisation strategies. Combined, the approach allows the entire
model to be specified using only a symbolic representation of model equations,
yields an efficient implementation of the forward model code, and enables the
automated derivation and efficient implementation of an associated discrete
adjoint model code.

Since the \verb+timestepping+ library builds upon the FEniCS system, it inherits
benefits associated with the application of automated code generation
techniques. In particular, automated code generation can be used to separate the
specification of a numerical model from the details of its implementation. The
\verb+timestepping+ library inherits much of the portability of the FEniCS
system: architectures and parallelisation strategies supported by the FEniCS
system can be used in combination with the \verb+timestepping+ library.

This manual describes the functionality and interfaces provided by the
\verb+timestepping+ Python library. This library makes extensive use
of the DOLFIN Python library, but the functionality and interfaces provided by
DOLFIN will not be described in detail here. For further details of DOLFIN and
the FEniCS system the reader is instead referred to \citet{logg2012}. In
addition, some familiarity with the Python language, the finite element method,
simple time discretisations, and adjoint modelling is assumed.

\section{General principles}\label{sect:principles}

The \verb+timestepping+ library works on the principle that a timestepping
model can be broken into three key stages:
\begin{enumerate}
  \item The initialisation stage: A series of equations which are solved
        exactly once.
  \item The timestepping stage: A series of equations which are solved
        repeatedly, and potentially a very large number of times.
  \item The finalisation stage: A series of equations which are solved
        exactly once.
\end{enumerate}
The library further assumes that the timestepping stage can be sub-divided into
two key stages:
\begin{enumerate}
  \item The timestep solve stage: A series of timestep equations are solved,
        using earlier time data as input and yielding future time data as
        output.
  \item The timestep variable cycle stage: Past time data is replaced using
        later time data.
\end{enumerate}
The interfaces provided by the library mimic this structure: the library
provides a syntax for the description of each of these key stages. The
\verb+timestepping+ library may not be suitable for problems which are
incompatible with this representation.

The implementation of the \verb+timestepping+ library makes a series of
additional assumptions regarding the performance demands of the timestepping
model. While these are not in principle essential to the development of time
discretisation specific optimisations, they have nevertheless generally been
applied in the development of the library. The assumptions are:
\begin{enumerate}
  \item Model initialisation is cheap.
  \item Work performed on every model timestep is expensive.
  \item Caching of data is preferable to recalculation.
\end{enumerate}
Since it is assumed that model initialisation is cheap, it is generally assumed
that the cost associated with the manipulation and analysis of model equations
is negligible. The timestepping model itself is optimised, but the process of
generating the optimised model is not. It is further generally assumed that
all work performed by a model timestep should be minimised, and therefore
that aggressive caching strategies should be applied. It is assumed that
sufficient memory is available for such caching\footnote{Storage of the forward
model solution, for the purposes of performing an adjoint calculation, is a
notable exception to this principle.}

\section{Dependencies}

The \verb+timestepping+ library has been tested with the following dependency
versions:

\begin{center}
\begin{tabular}{| c | c | }
\hline
Dependency  & Tested dependency versions \\
\hline
Python  & 3.5.2 \\
\hline
NumPy  & 1.11.0 \\
SciPy  & 0.17.0 \\
MPI4Py & 1.3.1 \\
VTK    & 8.0.1 \\
PETSc  & 3.7.7 \\
\hline
DOLFIN   & 2017.1.0 \\
FIAT     & 2017.1.0 \\
FFC      & 2017.1.0 \\
Instant  & 2017.1.0 \\
UFL      & 2017.1.0 \\
\hline
\end{tabular}
\end{center}

\section{Library interfaces}

The \verb+timestepping+ library can be divided into a primary high-level
interface and a secondary low-level interface. The high-level interface requires
interaction with high-level data types representing features such as model time
levels, functions, and model equations, and provides a syntax enabling the
description and implementation of a model with the structure discussed in
section \ref{sect:principles}. The low-level interface provides additional
advanced features, for example relating to the definition of explicit time
dependency, but requires interaction with the internal structure of the
\verb+timestepping+ library.

This manual documents only the primary high-level interface. Further
documentation relating to all functions and classes in the \verb+timestepping+
library can be accessed via the online documentation. For example, in a Python
shell:
\begin{lstlisting}
>>> import timestepping
>>> help(timestepping)
\end{lstlisting}
will display the online documentation.

\chapter{The high-level interface}

This chapter details the primary high-level interface provided by the
\verb+timestepping+ library. This interface provides a syntax enabling the
description and implementation of a model with the structure discussed in
section \ref{sect:principles}.

\section{Accessing the library}

The \verb+timestepping+ library is intended to be used in combination with
DOLFIN. The \verb+timestepping+ library overrides some aspects of the DOLFIN
interface, and hence the standard way of accessing the library is via:
\begin{lstlisting}
from dolfin import *
from timestepping import *
\end{lstlisting}

\section{Time level abstraction}

Model time levels are represented in the following way:
\begin{enumerate}
  \item Initialisation stage time levels: Via an integer or
        \verb+fractions.Fraction+.
  \item Timestep stage time levels: Via a \verb+TimeLevel+ object.
  \item Finalisation stage time levels: Via a \verb+FinalTimeLevel+ object.
\end{enumerate}

In the initialisation stage time levels are represented using an integer
or a \verb+fractions.Fraction+. The value zero is used to indicate the
transition point between the initialisation stage and the timestepping stage.

In the timestepping stage time levels are represented using a \verb+TimeLevel+
object. A \verb+TimeLevel+ has an associated integer or
\verb+fractions.Fraction+ ``offset'', enabling the logical comparison of
different \verb+TimeLevel+ objects. The \verb+timestepping+ library provides
an abstract handle \verb+n+, which is a \verb+TimeLevel+ assigned an
offset value of zero.

In the finalisation stage time levels are represented using a
\verb+FinalTimeLevel+ object. A \verb+FinalTimeLevel+ has an associated integer
or \verb+fractions.Fraction+ ``offset'' enabling the logical comparison of
different \linebreak \verb+FinalTimeLevel+ objects. The \verb+timestepping+ library
provides an abstract handle \verb+N+, which is a
\verb+FinalTimeLevel+ assigned an offset value of zero. An offset value of zero
is used to indicate the transition point between the timestepping stage and the
finalisation stage.

In the initialisation stage levels represented using an integer or
\verb+fractions.Fraction+ are well defined, but those represented using a
\verb+TimeLevel+ or \verb+FinalTimeLevel+ are not. Immediately after
initialisation levels represented using an integer, \verb+fractions.Fraction+,
or \verb+TimeLevel+ are well defined, but those represented using a
\verb+FinalTimeLevel+ are not. After the first timestep levels represented using
a \verb+TimeLevel+ are well defined, but those represented using an integer,
\verb+fractions.Fraction+, or \verb+FinalTimeLevel+ are not. After finalisation
levels represented using a \verb+TimeLevel+ or \verb+FinalTimeLevel+ are well
defined, but those represented using an integer or \verb+fractions.Fraction+ are
not.

Immediately after initialisation levels represented using an integer or
\verb+fractions.Fraction+ with value \verb+i+ are equivalent to
levels represented using a \verb+TimeLevel+ with value \verb=n + i=. After
finalisation levels represented using a \verb+TimeLevel+ with value \verb=n + i=
are equivalent to those represented using a \verb+FinalTimeLevel+
with value \verb=N + i=.

New \verb+TimeLevel+ and \verb+FinalTimeLevel+ objects can be instantiated
directly. The high-level constructor arguments are:
\begin{lstlisting}
  n_custom = TimeLevel(arg)
  N_custom = FinalTimeLevel(arg)
\end{lstlisting}
where the optional argument is an integer or \verb+fractions.Fraction+ defining
the offset. If this argument is not supplied then a default value of zero is
used. New \verb+TimeLevel+ and \verb+FinalTimeLevel+ objects can also be
instantiated via the addition or subtraction of an integer or
\verb+fractions.Fraction+ from an existing \verb+TimeLevel+ or
\verb+FinalTimeLevel+:
\begin{lstlisting}
n_custom = n + offset_1
N_custom = N - offset_2
\end{lstlisting}

\subsection*{Examples}
\begin{lstlisting}
from timestepping import *
from fractions import Fraction

# Create a TimeLevel with an offset of 2
n1 = TimeLevel(2)
# Create a TimeLevel with an offset of -1/2
n2 = TimeLevel(-Fraction(1, 2))
# Create a FinalTimeLevel with an offset of 0
N1 = FinalTimeLevel()

# Create a TimeLevel with an offset of 2
n3 = n + 2
# Create a TimeLevel with an offset of -3/2
n4 = n - Fraction(3, 2)

# Create a FinalTimeLevel with an offset of 1
N2 = N + 1

print(n1 == n2)  # False
print(n1 > n2)   # True
print(n1 == n3)  # True
print(n1 >= n3)  # True

print(N1 == N2)  # False
print(N1 <= N2)  # True
\end{lstlisting}

\section{Function time levels}

A \verb+TimeLevels+ object represents a unique set of model time levels,
combined with a ``cycle map'' defining how data is to be re-assigned during the
timestep variable cycle stage. A \verb+TimeLevels+ also defines the ``present''
point, defining which levels are to be treated as being in the past and which
levels are to be treated as being in the future.

A \verb+TimeLevels+ object is instantiated via:
\begin{lstlisting}
time_levels = TimeLevels(levels, cycle_map, last_past_level)
\end{lstlisting}
where \verb+levels+ is a list containing \verb+TimeLevel+ objects,
\verb+cycle_map+ is a dictionary with \verb+TimeLevel+ keys and values, and the
optional \verb+last_past_level+ is a \verb+TimeLevel+.

\verb+levels+ defines the model time levels which are to be represented by the
\verb+TimeLevels+ object. The elements of \verb+levels+ must be unique.

\verb+cycle_map+ defines the timestep variable cycle. At the end of a model
timestep data corresponding to each key in \verb+cycle_map+ is replaced with
data corresponding to the associated value in the \verb+cycle_map+. The
replacement of data is performed in order, with earlier (lower offset) keys in
\verb+cycle_map+ replaced before later keys. All keys in \verb+cycle_map+ must
be earlier (have a lower offset) than their associated values. The elements
iterated over via \verb+cycle_map.values()+ must be unique.

The optional \verb+last_past_level+ defines the ``present'' point associated
with the \verb+TimeLevels+. Time levels before or equal to
\verb+last_past_level+ (with a lower or equal offset) are treated as being in
the past, while other levels are treated as being in the future. If
\verb+last_past_level+ is not supplied then a default value of \verb+n+ is used.
All keys in \verb+cycle_map+ must be in the past.

\subsection*{Examples}

\begin{lstlisting}
from timestepping import *
from fractions import *

# Create a TimeLevels with one past level and one future level, with
# the past level data replaced by the future level data during the
# timestep variable cycle. Suitable for use with forward Euler,
# backward Euler, or Crank-Nicolson schemes.
levels_1 = TimeLevels(levels = [n, n + 1], cycle_map = {n:n + 1})

# Create a TimeLevels with one future level
levels_2 = TimeLevels(levels = [n], cycle_map = {},
  last_past_level = n - 1)

# Create a TimeLevels with one past level and two future levels,
# with the past level data replaced by the latest future level data
# during the timestep variable cycle. Suitable for use with a
# second order Runge-Kutta scheme.
levels_3 = TimeLevels(levels = [n, n + Fraction(1, 2), n + 1],
  cycle_map = {n:n + 1})

# Create a TimeLevels with three past levels and one future level,
# with the past levels replaced by the nearest later level during
# the timestep variable cycle. Suitable for use with a third order
# Adams-Bashforth scheme.
levels_4 = TimeLevels(levels = [n - 2, n - 1, n, n + 1],
  cycle_map = {n - 2:n - 1, n - 1:n, n:n + 1})
\end{lstlisting}

\section{Time dependent functions}

A \verb+TimeFunction+ object defines a series of \verb+dolfin.Function+ objects,
each specified on an individual time level.

A \verb+TimeFunction+ object is instantiated via:
\begin{lstlisting}
tfn = TimeFunction(levels, *args, **kwargs)
\end{lstlisting}
where \verb+levels+ is a \verb+TimeLevels+, \verb+args+ and \verb+kwargs+ are
arguments as accepted by a \verb+dolfin.Function+ constructor.

A symbolic representation of a single function, defined on a single time level,
can be accessed by indexing into the \verb+TimeFunction+ using an integer,
\verb+fractions.Fraction+, \verb+TimeLevel+, or \verb+FinalTimeLevel+. For
example:
\begin{lstlisting}
tfn[0]      # Representation of the function at time level 0
tfn[n + 1]  # Representation of the function at time level n + 1
tfn[N - 2]  # Representation of the function at time level N - 2
\end{lstlisting}
The index provided must match a level that, given the \verb+TimeLevels+ used to
instantiate the \verb+TimeFunction+, is well defined at some point in the model
execution.

Prior to initialisation a \verb+dolfin.Function+ returned by indexing into the
\verb+TimeFunction+ can be used for expressing equations symbolically, but
cannot be used to access model data. Immediately after initialisation the
\verb+dolfin.Function+ returned by indexing into the
\verb+TimeFunction+ with an integer, \verb+fractions.Fraction+ or
\verb+TimeLevel+ can be used to access model data, provided the time level is in
the past. After the first timestep the \verb+dolfin.Function+ returned by
indexing into the \verb+TimeFunction+ with a \verb+TimeLevel+ can be used to
access model data, provided the time level is in the past. After finalisation
the \verb+dolfin.Function+ returned by indexing into the \verb+TimeFunction+
with a \verb+TimeLevel+ or a \verb+FinalTimeLevel+ can be used to access model
data. In all other cases, the data associated with \verb+dolfin.Function+
objects returned by indexing into the \verb+TimeFunction+ are in an undefined
state.

\subsection*{Examples}

\begin{lstlisting}
from dolfin import *
from timestepping import *

# Define a simple structured mesh on the unit square
mesh = UnitSquareMesh(10, 10)
# P1 function space
space_p1 = FunctionSpace(mesh, "CG", 1)
# P2_{DG} function space
space_p2dg = FunctionSpace(mesh, "DG", 2)

# Define two different sets of time levels
levels_1 = TimeLevels(levels = [n, n + 1], cycle_map = {n:n + 1})
levels_2 = TimeLevels(levels = [n], cycle_map = {},
  last_past_level = n - 1)

# Define time dependent functions on the levels
F1 = TimeFunction(levels_1, space_p1, name = "F1")
F2 = TimeFunction(levels_2, space_p2dg, name = "F2")

# An equation involving the two time dependent functions
test_p1 = TestFunction(space_p1)
eq = inner(test_p1, F1[n + 1]) * dx == inner(test_p1, F2[n]) * dx
\end{lstlisting}

\section{Model equations}\label{sect:equations}

A \verb+TimeSystem+ object is used to keep a record of model equations.
Model equations can be registered with a \verb+TimeSystem+ by calling the
\verb+add_assignment+ or \verb+add_solve+ methods.

A \verb+TimeSystem+ is instantiated without arguments:
\begin{lstlisting}
system = TimeSystem()
\end{lstlisting}

\subsection{Registering assignments}\label{sect:assignments}

An assignment can be registered via the \verb+add_assignment+ method:
\begin{lstlisting}
system.add_assignment(y, x)
\end{lstlisting}
Here \verb+x+ is a \verb+dolfin.Function+ corresponding to a time level of a
\verb+TimeFunction+, and \verb+y+ is a float, a \linebreak \verb+dolfin.Constant+, a
\verb+dolfin.Function+, a \verb+LinearCombination+ (explained below), or a
general \linebreak \verb+ufl.core.expr.Expr+ (an arbitrary expression). This call
corresponds to an equation in which \verb+x+ is set equal to the value of
\verb+y+. The value of \verb+y+ must correspond to a either a constant value or
a function defined in the function space of \verb+x+. The assignment must also
be linear, with \verb+y+ independent of \verb+x+.

A \verb+LinearCombination+ is instantiated via:
\begin{lstlisting}
lc = LinearCombination(*args)
\end{lstlisting}
where the arguments \verb+args+ are an arbitrary list of tuples
\verb+(alpha, y)+. Each \verb+alpha+ is a float, a \verb+dolfin.Constant+, a
\verb+dolfin.Function+, or a general \verb+ufl.core.expr.Expr+. The
\verb+LinearCombination+ then represents the value of $\sum_i \alpha_i y_i$.
Note that the \verb+LinearCombination+ constructor does not check that all the
\verb+alpha+ are independent of all the \verb+y+, and consequently does not
check that the object represents a true linear combination. All the \verb+y+
must be in the same function space, and all the \verb+alpha+ must correspond
either to a constant value or a function defined in the function space of the
\verb+y+.

\subsection{Registering finite element variational problems}

An equation representing a finite element spatial discretisation can be
registered via the \verb+add_solve+ method. The arguments of this method are
largely based upon the arguments accepted by the \verb+dolfin.solve+ function.
The following are the high-level interfaces offered by this method:

\begin{lstlisting}
system.add_solve(y, x)
\end{lstlisting}
where \verb+y+ and \verb+x+ are as expected by the \verb+add_assignment+ method.
This calls the \verb+add_assignment+ method.

\begin{lstlisting}
system.add_solve(a == L, x, bcs,
  solver_parameters = solver_parameters,
  adjoint_solver_parameters = adjoint_solver_parameters)
\end{lstlisting}
where \verb+a+ is a rank 2 \verb+dolfin.Form+, \verb+L+ is a rank 1
\verb+dolfin.Form+, and \verb+x+ is a \verb+dolfin.Function+ corresponding to a
time level of a \verb+TimeFunction+. This registers a (possibly
non-linear) variational problem. Alternatively, one may use:
\begin{lstlisting}
system.add_solve(eq, x, bcs,
  solver_parameters = solver_parameters,
  adjoint_solver_parameters = adjoint_solver_parameters)
\end{lstlisting}
where \verb+eq+ is a (possibly non-linear) \verb+dolfin.Equation+ depending on
\verb+x+, and \verb+x+ is a \verb+dolfin.Function+ corresponding to a time level
of a \verb+TimeFunction+. This registers a (possibly non-linear) variational
problem represented by the equation \verb+eq+.

The optional \verb+bcs+ is a \verb+dolfin.DirichletBC+ or a list of
\verb+dolfin.DirichletBC+ objects, defining Dirichlet boundary conditions to be
applied when solving the variational problem. The optional
\verb+solver_parameters+ is a dictionary of DOLFIN solver parameters for use in
solving the variational problem, while the optional
\verb+adjoint_solver_parameters+ is a dictionary of DOLFIN linear solver
parameters for use in solving an associated adjoint variational problem. If
\verb+solver_parameters+ is not supplied then the default parameters defined in
\verb+dolfin.parameters+ (at the time of the \verb+TimeSystem.assemble+ call --
see section \ref{sect:assembly}) are used. If \verb+adjoint_solver_parameters+
is not supplied then the forward solver parameters are used in solving an
associated adjoint variational problem.

\subsection{Permitted equation dependencies}

The solution \verb+dolfin.Function+ of a registered equation, \verb+x+,
corresponds either to an initialisation stage time level, a timestep stage time
level, or a finalisation stage time level. Any time level dependencies of the
equation must be in this stage. For example, any time level dependencies of an
equation which solves for an initialisation stage time level must themselves
correspond to an initialisation stage time level. The following is a valid
registration:
\begin{lstlisting}
system.add_solve(F[-1], F[0])
\end{lstlisting}
(provide that \verb+F+ is defined on the relevant levels), while the following
is an \emph{invalid} registration:
\begin{lstlisting}
system.add_solve(F[n], F[0])
\end{lstlisting}

If the solution \verb+dolfin.Function+ of a registered equation, \verb+x+,
corresponds to an initialisation stage time level, then the time level must be
in the past. If \verb+x+ corresponds to a timestep stage or finalisation stage
time level, then the time level must be in the future. No dependencies of a
registered equation can correspond to time levels later than \verb+x+.

Note that the order of equation registration is irrelevant, and that
circular dependencies are not permitted -- dependency resolution is applied in
the analysis and optimisation step (see section \ref{sect:assembly}).

\subsection*{Examples}

\begin{lstlisting}
from dolfin import *
from timestepping import *

# Define a simple structured mesh on the unit interval
mesh = UnitIntervalMesh(10)
# P1 function space
space = FunctionSpace(mesh, "CG", 1)

# Model parameters and boundary conditions
dt = Constant(0.05)
bc1 = DirichletBC(space, 1.0, "on_boundary && near(x[0], 0.0)")
bc2 = DirichletBC(space, 0.0, "on_boundary && near(x[0], 1.0)")
bcs = [bc1, bc2]
nu = Constant(0.01)

# Define time levels
levels = TimeLevels(levels = [n, n + 1], cycle_map = {n:n + 1})
# A time dependent function
u = TimeFunction(levels, space, name = "u")

# Initialise a TimeSystem
system = TimeSystem()

# Add an initial assignment
u_ic = Function(space, name = "u_ic")
u_ic.assign(Constant(0.0))
bc1.apply(u_ic.vector())
system.add_solve(u_ic, u[0])
# Register a simple diffusion equation, discretised in time
# using forward Euler
test = TestFunction(space)
system.add_solve(
  inner(test, (1.0 / dt) * (u[n + 1] - u[n])) * dx ==
    -nu * inner(grad(test), grad(u[n])) * dx,
  u[n + 1], bcs,
  solver_parameters = {"linear_solver":"lu"})
\end{lstlisting}
See also section \ref{sect:statics} for a modification of this example,
declaring static data.

\section{Static data}\label{sect:statics}

Many optimisations of the timestepping model require the identification of
static data, which does not vary throughout the model execution. Additional
functions are provided to facilitate such optimisations.

The \verb+StaticConstant+ function returns a \verb+dolfin.Constant+ that is
declared as ``static'. The arguments are identical to the DOLFIN \verb+Constant+
constructor. The \verb+StaticFunction+ function returns a \verb+dolfin.Function+
that is declared as ``static'. The arguments are identical to the DOLFIN
\verb+Function+ constructor . The \linebreak \verb+StaticDirichletBC+ class defines a
\verb+dolfin.DirichletBC+ that is declared as ``static''. The constructor
arguments are identical to the \verb+dolfin.DirichletBC+ constructor.

Objects that are declared as ``static'' in this way should not be modified
during the model execution -- doing so leads to undefined behaviour. Objects
declared as static may be freely modified before the analysis and optimisation
step (see section \ref{sect:assembly}), and functionality is provided to enable
the safe modification of static \verb+Constant+ and \verb+Function+ objects
after the finalisation stage (section \ref{sect:reassembly}).

\subsection*{Examples}

\begin{lstlisting}
from dolfin import *
from timestepping import *

# Create a constant which is declared as static
c = StaticConstant(1.0, name = "c")

# Define a simple structured mesh on the unit interval
mesh = UnitIntervalMesh(10)
# P1 function space
space = FunctionSpace(mesh, "CG", 1)
# Create a function which is declared as static
F = StaticFunction(space, name = "F")
# Initialise the static function
F.interpolate(Expression("x[0]", element = F.function_space().ufl_element()))

# Create a Dirichlet boundary condition which is declared as static
bc = StaticDirichletBC(space, 0.0, "on_boundary")
\end{lstlisting}

\begin{lstlisting}
from dolfin import *
from timestepping import *

# Define a simple structured mesh on the unit interval
mesh = UnitIntervalMesh(10)
# P1 function space
space = FunctionSpace(mesh, "CG", 1)

# Model parameters and boundary conditions
dt = StaticConstant(0.05)
bc1 = StaticDirichletBC(space, 1.0,
  "on_boundary && near(x[0], 0.0)")
bc2 = StaticDirichletBC(space, 0.0,
  "on_boundary && near(x[0], 1.0)")
bcs = [bc1, bc2]
nu = StaticConstant(0.01)

# Define time levels
levels = TimeLevels(levels = [n, n + 1], cycle_map = {n:n + 1})
# A time dependent function
u = TimeFunction(levels, space, name = "u")

# Initialise a TimeSystem
system = TimeSystem()

# Add an initial assignment
u_ic = StaticFunction(space, name = "u_ic")
u_ic.assign(Constant(0.0))
bc1.apply(u_ic.vector())
system.add_solve(u_ic, u[0])
# Register a simple diffusion equation, discretised in time
# using forward Euler
test = TestFunction(space)
system.add_solve(
  inner(test, (1.0 / dt) * (u[n + 1] - u[n])) * dx ==
    -nu * inner(grad(test), grad(u[n])) * dx,
  u[n + 1], bcs,
  solver_parameters = {"linear_solver":"lu"})
\end{lstlisting}

\section{Analysis and optimisation}\label{sect:assembly}

The \verb+assemble+ method of a \verb+TimeSystem+ analyses the model, performs
time discretisation specific optimisations, and (by default) performs the
initialisation stage, returning an object suitable for execution of the model
time loop. An adjoint model is enabled by supplying additional arguments to the
\verb+assemble+ method.

The \verb+assemble+ method returns a \verb+ForwardModel+ if no adjoint model is
enabled, and an \verb+ManagedModel+ otherwise. Note that there is a fairly
strong assumption that at most one \verb+ForwardModel+ or
\verb+ManagedModel+ is instantiated at once. Instantiating two or more at the
same time leads to undefined behaviour.

\subsection{Forward model only}\label{sect:forward_assembly}

If an adjoint model is not enabled, the \verb+assemble+ method has high-level
arguments:
\begin{lstlisting}
asystem = system.assemble(initialise = initialise)
\end{lstlisting}
This returns a \verb+ForwardModel+. If the optional \verb+initialise+ is not
supplied or is \verb+True+ then, immediately after the \verb+assemble+ call, the
initialisation stage is complete. If it is \verb+False+ the initialisation stage
can be performed via a subsequent call to the \verb+initialise+ method:
\begin{lstlisting}
asystem.initialise()
\end{lstlisting}

\subsection{Enabling an adjoint model}\label{sect:adjoint_assembly}

An adjoint model is enabled by calling the \verb+assemble+ method with the
optional \verb+adjoint+ boolean argument set equal to \verb+True+. The remaining
high-level arguments are:
\begin{lstlisting}
asystem = system.assemble(adjoint = adjoint,
  disk_period = disk_period, functional = functional,
  initialise = initialise)
\end{lstlisting}
This returns an \verb+ManagedModel+. If the optional \verb+initialise+ is not
supplied or is \verb+True+ then, immediately after the \verb+assemble+ call, the
initialisation stage is complete. Otherwise, the initialisation stage can be
performed via a subsequent call to the \verb+initialise+ method:
\begin{lstlisting}
asystem.initialise()
\end{lstlisting}
The optional integer \verb+disk_period+ defines how often the forward model
solution is to be checkpointed to disk. Data is checkpointed every
\verb+disk_period+ timesteps, and is stored in a directory \verb+checkpoints~+
relative to the current working directory. If \verb+disk_period+ is not supplied
then the entire forward model solution is stored in memory. In the high-level
interface the optional \verb+functional+ is a rank 0 form.
\verb+dolfin.Function+ dependencies of the functional which correspond to levels
of a \verb+TimeFunction+ must correspond to finalisation stage time levels. If
\verb+functional+ is not supplied to the \verb+assemble+ method then a
functional can instead be specified via the \verb+set_functional+ method:
\begin{lstlisting}
asystem.set_functional(functional)
\end{lstlisting}

\subsection{Modifying static data}\label{sect:reassembly}

In some circumstances it is useful to modify the values of static \verb+Constant+
or \verb+Function+ objects after the analysis and optimisation. This may be
useful, for example, after the completion of forward model timestepping, in
order to re-run the forward model with a new value for a parameter. However in
this case the data cached during the analysis and optimisation step may become
invalid. This issue can be resolved by calling the \verb+reassemble+ method:
\begin{lstlisting}
asystem.reassemble(*args)
\end{lstlisting}
If supplied with no arguments then this method regenerates all cached data.
Optionally an arbitrary number of \verb+Constant+ or \verb+Function+ arguments
may be supplied, indicating that only data associated with these objects should
be regenerated.

\section{Timestepping}\label{sect:timestepping}

The model is timestepped by calling the \verb+timestep+ method of a
\verb+ForwardModel+ or \verb+ManagedModel+:
\begin{lstlisting}
asystem.timestep(s = s)
\end{lstlisting}
where the optional integer \verb+s+ is the number of timesteps to perform. If
\verb+s+ is not supplied then a single timestep is performed. The
\verb+timestep+ method performs both the timestep solve and timestep cycle
stages. At the model end the finalisation stage is performed by calling the
\verb+finalise+ method:
\begin{lstlisting}
asystem.finalise()
\end{lstlisting}

The initialisation stage should be complete before calling the \verb+timestep+
method (see section \ref{sect:assembly}). If an adjoint model is not enabled
then the model can, after a \verb+finalise+ call, be restarted via a call to the
\verb+initialise+ method (see section \ref{sect:forward_assembly}). In this case
the \verb+timestep+ and \verb+finalise+ methods should not be called between the
\verb+finalise+ and \verb+initialise+ calls. If the model is not restarted, or
if an adjoint model is enabled, then the \verb+timestep+ and \verb+finalise+
methods should not be called after the \verb+finalise+ call.

\subsection*{Examples}

\begin{lstlisting}
from dolfin import *
from timestepping import *

# Define a simple structured mesh on the unit interval
mesh = UnitIntervalMesh(10)
# P1 function space
space = FunctionSpace(mesh, "CG", 1)

# Model parameters and boundary conditions
dt = StaticConstant(0.05)
bc1 = StaticDirichletBC(space, 1.0,
  "on_boundary && near(x[0], 0.0)")
bc2 = StaticDirichletBC(space, 0.0,
  "on_boundary && near(x[0], 1.0)")
bcs = [bc1, bc2]
nu = StaticConstant(0.01)

# Define time levels
levels = TimeLevels(levels = [n, n + 1], cycle_map = {n:n + 1})
# A time dependent function
u = TimeFunction(levels, space, name = "u")

# Initialise a TimeSystem
system = TimeSystem()

# Add an initial assignment
u_ic = StaticFunction(space, name = "u_ic")
u_ic.assign(Constant(0.0))
bc1.apply(u_ic.vector())
system.add_solve(u_ic, u[0])
# Register a simple diffusion equation, discretised in time
# using forward Euler
test = TestFunction(space)
system.add_solve(
  inner(test, (1.0 / dt) * (u[n + 1] - u[n])) * dx ==
    -nu * inner(grad(test), grad(u[n])) * dx,
  u[n + 1], bcs,
  solver_parameters = {"linear_solver":"lu"})

# Assemble the TimeSystem
system = system.assemble()

# Timestep the model
t = 0.0
while t * (1.0 + 1.0e-9) < 1.0:
  system.timestep()
  t += float(dt)
# Finalise
system.finalise()
\end{lstlisting}

\section{Adjoint modelling}

Given a forward model described using a \verb+TimeSystem+, the associated
discrete adjoint model can be derived in the analysis and optimisation step (see
section \ref{sect:adjoint_assembly}). The resulting \verb+ManagedModel+
returned can be used to perform a total derivative calculation.

\subsection{Checkpointing and storage}

An adjoint model calculation requires knowledge of the forward model solution
and, as a result, forward model data must be stored. It is often infeasible to
store all required data in memory, and hence disk checkpointing and recovery
must be used. In this procedure forward model data, sufficient to restart the
forward model, are checkpointed periodically, and forward model data required
between checkpoints are re-computed and stored. Forward model checkpointing and
storage is configured in the analysis and optimisation step (see section
\ref{sect:adjoint_assembly}).

After the forward model finalisation stage (after the
\verb+ManagedModel.finalise+ call) stored forward model data can be verified via
the \verb+verify_checkpoints+ method:
\begin{lstlisting}
asystem.verify_checkpoints(tolerance = tolerance)
\end{lstlisting}
where the optional \verb+tolerance+ argument is a positive float defining the
tolerance used in comparisons. The \verb+verify_checkpoints+ method raises an
exception on failure. For serial calculations the forward model data should be
exactly reproducible, and verification should succeed with a tolerance of zero.
For parallel calculations the verification may fail with a zero or small value
for the tolerance. Note that the \verb+verify_checkpoints+ method re-runs the
entire forward model, and hence this should only be used for debugging purposes.

\subsection{Derivative calculation}

A total derivative can be computed via the \verb+compute_gradient+ method. The
high-level arguments for this method are:
\begin{lstlisting}
grad = asystem.compute_gradient(parameters = parameters,
  project = project,
  project_solver_parameters = project_solver_parameters)
\end{lstlisting}
where \verb+parameters+ is a \verb+dolfin.Constant+, a \verb+dolfin.Function+,
or a list of \verb+dolfin.Constant+ or \verb+dolfin.Function+ objects, the
optional \verb+project+ is a boolean, and the optional
\verb+project_solver_parameters+ is a dictionary of linear solver parameters. If
\verb+parameters+ is a \verb+dolfin.Constant+ or \verb+dolfin.Function+ then it
must have been declared as static (see section \ref{sect:statics}). If
\verb+parameters+ is a list of \verb+dolfin.Constant+ or \verb+dolfin.Function+
objects then all elements must have been declared as static. This method returns
the total derivative of the registered functional (see section
\ref{sect:adjoint_assembly}). If \verb+parameters+ is a list, then the
\verb+i+th element of the return value contains the derivative with respect to
the \verb+i+th element of \verb+parameters+. The derivative with respect to a
\verb+Constant+ parameter is returned as a \verb+Constant+. If \verb+project+ is
\verb+False+ (the default) then the derivative with respect to a \verb+Function+
parameter is returned as a \verb+GenericVector+. If \verb+project+ is
\verb+True+ then the derivative with respect to a \verb+Function+ parameter is
returned as a \verb+(GenericVector, Function)+ pair, where the \verb+Function+
contains the derivative projected onto the parameter trial space (via a least
squares projection), and where \verb+project_solver_parameters+ defines linear
solver options for the associated mass matrix inversion.

The \verb+compute_gradient+ method must be called after the forward model
finalisation stage (after the \linebreak \verb+ManagedModel.finalise+ call), and a
functional must be registered.

\subsection{Adjoint verification}

Let $\vec{m}$ correspond to the degrees of freedom associated with a parameter,
and define a functional
$J \left( \vec{x} \left( \vec{m} \right), \vec{m} \right)$, where $\vec{x}$
corresponds to the degrees of freedom associated with the forward model solution.
Then:
\begin{equation}
  \left| J \left( \vec{x} \left( \vec{m} + \epsilon \vec{\delta m} \right), \vec{m} + \epsilon \vec{\delta m} \right)
    - J \left( \vec{x} \left( \vec{m} \right), \vec{m} \right) \right|
    = {\cal O} \left( \epsilon \right),
\end{equation}
converges asymptotically at first order in perturbations of the parameter
$\vec{m}$. However:
\begin{equation}
  \left| J \left( \vec{x} \left( \vec{m} + \epsilon \vec{\delta m} \right), \vec{m} + \epsilon \vec{\delta m} \right)
    - J \left( \vec{x} \left( \vec{m} \right), \vec{m} \right)
    - \epsilon \left< \frac{d J}{d \vec{m}}, \vec{\delta m} \right> \right|
    = {\cal O} \left( \epsilon^2 \right),
\end{equation}
converges asymptotically at second order. Hence the result of a model
constrained derivative calculation, yielding $d J / d \vec{m}$, can be verified
using repeated forward model calculations, each with a small perturbation of the
parameter $\vec{m}$.

The \verb+taylor_test+ method performs such a verification. The high-level
arguments for this method are:
\begin{lstlisting}
orders = asystem.taylor_test(parameter, grad = grad,
  ntest = ntest, fact  = fact)
\end{lstlisting}
\verb+parameter+ is a \verb+dolfin.Constant+ or \verb+dolfin.Function+ that has
been declared as static (see section \ref{sect:statics}) and the optional
\verb+grad+ is the corresponding derivative returned by the
\verb+compute_gradient+ method. If \verb+grad+ is not supplied then the
total derivative is computed using the adjoint model. The optional \verb+ntest+
argument is the number of forward model calculations to perform in the
verification. This must be greater than or equal to 2, and a default value of 6
is used if this is not supplied. The optional \verb+fact+ is a positive float
and sets the magnitude of the perturbations. Let $N$ be the value of
\verb+ntest+ and \verb+f+ be the value of \verb+fact+. If \verb+parameter+ is a
\verb+dolfin.Constant+ with value $m$ then the perturbations applied are
$\epsilon_i \delta m$, where:
\begin{subequations}
  \begin{equation}
    \epsilon_i = 2^{N - i}, \quad i \in \left\{1, \ldots, N \right\},
  \end{equation}
  \begin{equation}
    \delta m = \left\{ \begin{array}{l} f \left| m \right| \quad \textrm{if } \left| m \right| > 0 \\ f \qquad \quad \textrm{otherwise} \end{array} \right. .
  \end{equation}
\end{subequations}
If \verb+parameter+ is a \verb+dolfin.Function+ with degrees of freedom
with associated values $\vec{m}$ then the perturbations applies are
$\epsilon_i \vec{\delta m}$, where:
\begin{subequations}
  \begin{equation}
    \epsilon_i = 2^{N - i}, \quad i \in \left\{1, \ldots, N \right\},
  \end{equation}
  \begin{equation}
    \vec{\delta m} = \left\{ \begin{array}{l} \vec{F} \left|\left| \vec{m} \right|\right|_\infty \quad \textrm{if } \left|\left| \vec{m} \right|\right|_\infty > 0 \\ \vec{F} \qquad \qquad \textrm{otherwise} \end{array} \right.
  \end{equation}
\end{subequations}
where the values of $\vec{F}$ are randomly generated values in the range
$\left[ -f/2, +f/2 \right)$, with a uniform probability distribution. If
\verb+fact+ is not supplied then a default value of $10^{-4}$ is used.

The \verb+taylor_test+ method returns a list of relative orders of
convergence. The elements of the returned list have values:
\begin{equation}
  o_i = \textrm{log}_2 \frac{r_{i + 1}}{r_i},
\end{equation}
where:
\begin{subequations}
  \begin{equation}
    \epsilon_i = 2^{N - i}, \quad i \in \left\{1, \ldots, N \right\},
  \end{equation}
  \begin{equation}
    r_i = \left| J \left( \vec{x} \left( \vec{m} + \epsilon_i \vec{\delta m} \right), \vec{m} + \epsilon_i \vec{\delta m} \right)
      - J \left( \vec{x} \left( \vec{m} \right), \vec{m} \right)
      - \epsilon_i \left< \frac{d J}{d \vec{m}}, \vec{\delta m} \right> \right|.
  \end{equation}
\end{subequations}
The elements of the returned list should be close to 2 in a successful
verification.

It is recommended that new adjoint models should be verified using a Taylor
remainder convergence test. Since the test involves repeated running of the
forward model it may be necessary to test the adjoint on a small problem,
for example using a reduced number of model timesteps. The Taylor remainder
convergence test may fail if:
\begin{enumerate}
  \item The perturbations are too large, meaning that the asymptotic convergence
        of the remainder is not observed.
  \item The perturbations are too small, meaning that the convergence test is
        polluted by machine precision errors.
  \item The forward model is not differentiable, meaning that a model
        constrained derivative does not exists.
  \item Iterative solver tolerances are too large.
  \item The adjoint model is inconsistent with the forward model.
\end{enumerate}
The first two cases can be avoided by decreasing or increasing \verb+fact+
(assuming that these two cases do not both apply simultaneously, in which
case a Taylor remainder convergence test cannot be used). The final case may
occur if forward model data is modified manually during the model execution, or
may indicate a bug in the \verb+timestepping+ library.

\subsection*{Examples}

\begin{lstlisting}
from dolfin import *
from timestepping import *

# Define a simple structured mesh on the unit interval
mesh = UnitIntervalMesh(10)
# P1 function space
space = FunctionSpace(mesh, "CG", 1)

# Model parameters and boundary conditions
dt = StaticConstant(0.05)
bc1 = StaticDirichletBC(space, 1.0,
  "on_boundary && near(x[0], 0.0)")
bc2 = StaticDirichletBC(space, 0.0,
  "on_boundary && near(x[0], 1.0)")
bcs = [bc1, bc2]
nu = StaticConstant(0.01)

# Define time levels
levels = TimeLevels(levels = [n, n + 1], cycle_map = {n:n + 1})
# A time dependent function
u = TimeFunction(levels, space, name = "u")

# Initialise a TimeSystem
system = TimeSystem()

# Add an initial assignment
u_ic = StaticFunction(space, name = "u_ic")
u_ic.assign(Constant(0.0))
bc1.apply(u_ic.vector())
system.add_solve(u_ic, u[0])
# Register a simple diffusion equation, discretised in time
# using forward Euler
test = TestFunction(space)
system.add_solve(
  inner(test, (1.0 / dt) * (u[n + 1] - u[n])) * dx ==
    -nu * inner(grad(test), grad(u[n])) * dx,
  u[n + 1], bcs,
  solver_parameters = {"linear_solver":"lu"})

# Assemble the TimeSystem, enabling the adjoint. Set the
# functional to be equal to spatial integral of the final u.
system = system.assemble(adjoint = True, functional = u[N] * dx)

# Timestep the model
t = 0.0
while t * (1.0 + 1.0e-9) < 1.0:
  system.timestep()
  t += float(dt)
# Finalise
system.finalise()

# Perform a total derivative calculation
dJ = system.compute_gradient(nu)

# Verify the stored forward model data
system.verify_checkpoints()
# Verify the computed derivative using a Taylor remainder
# convergence test
orders = system.taylor_test(nu, grad = dJ)
# Check the convergence order
assert((orders > 1.99).all())
\end{lstlisting}

\chapter{Integration with \texttt{dolfin-adjoint}}

The \verb+dolfin-adjoint+ library constructs the discrete adjoints of models
written using the FEniCS system. Specifically, given a model written as a series
of discrete variational problems using the Python DOLFIN interface,
\verb+dolfin-adjoint+ constructs a ``tape'' of model equations and uses this
tape to derive and implement discrete adjoint models \citep{farrell2013}.
\verb+dolfin-adjoint+ provides a number of advanced features, including:
\begin{enumerate}
  \item Automated derivation of tangent-linear, adjoint, and second order
        adjoint models.
  \item Gradient calculations, with automated Taylor verification.
  \item Hessian action calculations, with automated Taylor verification. 
  \item Optimal checkpointing via the \verb+resolve+ library
        \citep{griewank2000}.
  \item Tools for performing principle component analyses.
  \item Parallel optimisation (functional minimisation), including both gradient
        and Hessian based optimisation \citep{funke2013}.
\end{enumerate}

The \verb+timestepping+ library can be used in combination with the
\verb+dolfin-adjoint+ library. This enables a transient model to be written
using a high-level representation, and combines the application of time
discretisation specific optimisations with the more advanced features available
in \verb+dolfin-adjoint+.

For full details of the \verb+dolfin-adjoint+ library, see the
\verb+dolfin-adjoint+ documentation \citep{da}.

\section{Accessing the libraries}

The combination of the \verb+timestepping+ and \verb+dolfin-adjoint+ libraries
may be accessed via:
\begin{lstlisting}
from dolfin import *
from dolfin_adjoint_timestepping import *
\end{lstlisting}

\section{Using \texttt{dolfin-adjoint}}

When using \verb+dolfin-adjoint+ with the \verb+timestepping+ library, the
model equations are registered as normal (see section \ref{sect:equations}),
and the forward model is executed as normal (section \ref{sect:timestepping}).
\verb+dolfin-adjoint+ functionality may be accessed upon completion of the
model timestepping (after the \verb+finalise+ call). Typically the annotation
of equations by \verb+dolfin-adjoint+ should be disabled upon completion of the
model timestepping:
\begin{lstlisting}
asystem.finalise()
parameters["adjoint"]["stop_annotating"] = True
\end{lstlisting}

\subsection*{Examples}

\begin{lstlisting}
from dolfin import *
from dolfin_adjoint_timestepping import *

### Stage 1: Configure and execute the forward model using
###          functionality provided by the timestepping library

# Define a simple structured mesh on the unit interval
mesh = UnitIntervalMesh(10)
# P1 function space
space = FunctionSpace(mesh, "CG", 1)

# Model parameters and boundary conditions
dt = StaticConstant(0.05)
bc1 = StaticDirichletBC(space, 1.0,
  "on_boundary && near(x[0], 0.0)")
bc2 = StaticDirichletBC(space, 0.0,
  "on_boundary && near(x[0], 1.0)")
bcs = [bc1, bc2]
nu = StaticConstant(0.01)

# Define time levels
levels = TimeLevels(levels = [n, n + 1], cycle_map = {n:n + 1})
# A time dependent function
u = TimeFunction(levels, space, name = "u")

# Initialise a TimeSystem
system = TimeSystem()

# Add an initial assignment
u_ic = StaticFunction(space, name = "u_ic")
u_ic.assign(Constant(0.0))
bc1.apply(u_ic.vector())
system.add_solve(u_ic, u[0])
# Register a simple diffusion equation, discretised in time
# using forward Euler
test = TestFunction(space)
system.add_solve(
  inner(test, (1.0 / dt) * (u[n + 1] - u[n])) * dx ==
    -nu * inner(grad(test), grad(u[n])) * dx,
  u[n + 1], bcs,
  solver_parameters = {"linear_solver":"lu"})

# Assemble the TimeSystem
system = system.assemble(initialise = False)

# Run the forward model. The model execution is wrapped by a
# function to enable adjoint verification using the
# dolfin-adjoint taylor_test function.
def run_forward():
  system.initialise()
  t = 0.0
  while t * (1.0 + 1.0e-9) < 1.0:
    system.timestep()
    t += float(dt)
  system.finalise()
  return
run_forward()

### Stage 2: Access features provided by the dolfin-adjoint library

# Disable annotation of model equations by dolfin-adjoint
parameters["adjoint"]["stop_annotating"] = True

# Define a functional equal to spatial integral of the final u
J = u[N] * dx
# Perform a total derivative calculation
J_da = Functional(J * dolfin_adjoint.dt[FINISH_TIME])
nu_da = Control(nu)
dJ = compute_gradient(J_da, nu_da)

# Verify the computed derivative using a Taylor remainder
# convergence test  
def J_p(nu_p):
  nu.assign(nu_p)
  system.reassemble(nu)
  run_forward()
  return assemble(J)
order = taylor_test(J_p, nu_da, assemble(J), dJ, seed = 1.0e-6)
# Check the convergence order
assert(order > 1.99)
\end{lstlisting}

\chapter{Time discretisation examples}

This chapter describes the implementation of a number of simple time
discretisations using the \verb+timestepping+ library. The provided examples are
largely illustrative, and may not be stable. The quoted orders of accuracy
assume that the scheme is stable for the problem considered.

All examples in this chapter consider
the discretisation of the following system:
\begin{subequations}
  \begin{equation}\label{eqn:t_timestep}
    \partial_t T = F(T),
  \end{equation}
  \begin{equation}\label{eqn:t_init}
    \left. T \right|_{t = 0} = T_0.
  \end{equation}
\end{subequations}
It is assumed that $T_0$ is defined using a DOLFIN variable \verb+T_0+, which
may for example be a \verb+dolfin.Constant+ or \verb+dolfin.Function+. It is
further assumed that some Python function for approximating
$\int_\Omega \phi_i F(T)$ over the space $\Omega$ and with test functions
$\phi_i$ is provided, with the following form:
\begin{lstlisting}
def F(test, T):
  rhs = ... # Insert definition here
  return rhs
\end{lstlisting}
where \verb+test+ is a spatial test function.

\section{Adams-Bashforth}

Adams-Bashforth schemes take the form:
\begin{equation}
  T^{n + 1} = T^n + \Delta t \sum_{i = 1}^N \gamma_i F( T^{n + 1 - i}),
\end{equation}
where the $\gamma_i$ are constants. Adams-Bashforth schemes therefore require
the values of $F(T)$ from $N$ previous time levels. Given a choice of $N$, the
$\gamma_i$ are chosen so as to yield $N$th order accuracy. If $N > 1$ then the
scheme is not self starting, and care is required when initialising in order to
maintain the accuracy of the method. Further details of Adams-Bashforth schemes
can be found in \citet[section 2.1]{iserles2009} and
\citet[section III.1]{hairer1987}.

\subsection{Forward Euler}\label{sect:fe}

For $N = 1$ the single coefficient $\gamma_1$ is chosen to be:
\begin{equation}
  \gamma_1 = 1,
\end{equation}
yielding:
\begin{equation}
  T^{n + 1} = T^n + \Delta t F(T^n).
\end{equation}
This is the forward Euler scheme, and has first order accuracy.

This discretisation can be implemented via:
\begin{lstlisting}
levels    = TimeLevels(levels = [n, n + 1], cycle_map = {n:n + 1})
levels_dT = TimeLevels(levels = [n], cycle_map = {},
  last_past_level = n - 1)
T  = TimeFunction(levels,    space, name = "T")
dT = TimeFunction(levels_dT, space, name = "dT")

system = TimeSystem()

system.add_solve(T_ic, T[0])

system.add_solve(inner(test, dT[n]) * dx == dt * F(test, T[n]),
  dT[n], solver_parameters = solver_parameters)
system.add_solve(LinearCombination((1.0, T[n]), (1.0, dT[n])),
  T[n + 1])
\end{lstlisting}
where \verb+space+ defines the function space for $T$, \verb+dt+ defines the
timestep size, and \verb+solver_parameters+ is a dictionary of solver
parameters.

\subsection{Second order Adams-Bashforth}\label{sect:ab2}

For $N = 2$ the $\gamma_i$ are chosen to be:
\begin{align}
  \gamma_1 = \frac{3}{2}, & & \gamma_2 = -\frac{1}{2},
\end{align}
yielding:
\begin{equation}
  T^{n + 1} = T^n + \Delta t \left[ \frac{3}{2} F(T^n) - \frac{1}{2} F(T^{n - 1}) \right].
\end{equation}
This scheme cannot be used to perform the first model timestep as $T^{-1}$ is
not available. A single forward Euler step \ref{sect:fe} can be applied:
\begin{equation}
  T^1 = T^0 + \Delta t F(T^0).
\end{equation}

This discretisation can be implemented via:
\begin{lstlisting}
levels    = TimeLevels(levels = [n, n + 1, n + 2],
  cycle_map = {n:n + 1, n + 1:n + 2}, last_past_level = n + 1)
levels_dT = TimeLevels(levels = [n, n + 1], cycle_map = {n:n + 1},
  last_past_level = n)
T  = TimeFunction(levels,    space, name = "T")
dT = TimeFunction(levels_dT, space, name = "dT")

system = TimeSystem()

system.add_solve(T_ic, T[0])

system.add_solve(inner(test, dT[0]) * dx == dt * F(test, T[0]),
  dT[0], solver_parameters = solver_parameters)
system.add_solve(LinearCombination((1.0, T[0]),
                                   (1.0, dT[0])),
  T[1])

system.add_solve(inner(test, dT[n + 1]) * dx ==
  dt * F(test, T[n + 1]),
  dT[n + 1], solver_parameters = solver_parameters)
system.add_solve(LinearCombination((1.0, T[n + 1]),
                                   (3.0 / 2.0, dT[n + 1]),
                                   (-1.0 / 2.0, dT[n])),
  T[n + 2])
\end{lstlisting}
At the end of the simulation the final model solution is stored in
\verb=T[N + 1]=.

\subsection{Third order Adams-Bashforth}

For $N = 3$ the $\gamma_i$ are chosen to be:
\begin{align}
  \gamma_1 = \frac{23}{12}, & & \gamma_2 = -\frac{4}{3}, & & \gamma_3 = \frac{5}{12},
\end{align}
yielding:
\begin{equation}
  T^{n + 1} = T^n + \Delta t \left[ \frac{23}{12} F(T^n) - \frac{4}{3} F(T^{n - 1}) + \frac{5}{12} F(T^{n - 2}) \right].
\end{equation}
This scheme cannot be used to perform the first two model timesteps as $T^{-1}$
and $T^{-2}$ are not available. A second order Runge-Kutta
\ref{sect:rk2} and a second order Adams-Bashforth \ref{sect:ab2} steps can be
applied:
\begin{subequations}
  \begin{equation}
    T^{\frac{1}{2}} = T^0 + \frac{1}{2} \Delta t F(T^0).
  \end{equation}
  \begin{equation}
    T^1 = T^0 + \Delta t F(T^{\frac{1}{2}}).
  \end{equation}
  \begin{equation}
    T^2 = T^1 + \Delta t \left[ \frac{3}{2} F(T^1) - \frac{1}{2} F(T^0) \right].
  \end{equation}
\end{subequations}

This discretisation can be implemented via:
\begin{lstlisting}
from fractions import Fraction
hf = Fraction(1, 2)

levels    = TimeLevels(levels = [n, n + hf, n + 1, n + 2, n + 3],
  cycle_map = {n:n + 1, n + 1:n + 2, n + 2:n + 3},
  last_past_level = n + 2)
levels_dT = TimeLevels(levels = [n, n + hf, n + 1, n + 2],
  cycle_map = {n:n + 1, n + 1:n + 2}, last_past_level = n + 1)
T  = TimeFunction(levels,    space, name = "T")
dT = TimeFunction(levels_dT, space, name = "dT")

system = TimeSystem()

system.add_solve(T_ic, T[0])

system.add_solve(inner(test, dT[0]) * dx == dt * F(test, T[0]),
  dT[0], solver_parameters = solver_parameters)
system.add_solve(LinearCombination((1.0, T[0]),
                                   (0.5, dT[0])),
  T[hf])

system.add_solve(inner(test, dT[hf]) * dx == dt * F(test, T[hf]),
  dT[hf], solver_parameters = solver_parameters)
system.add_solve(LinearCombination((1.0, T[0]),
                                   (1.0, dT[hf])),
  T[1])

system.add_solve(inner(test, dT[1]) * dx == dt * F(test, T[1]),
  dT[1], solver_parameters = solver_parameters)
system.add_solve(LinearCombination((1.0, T[1]),
                                   (3.0 / 2.0, dT[1]),
                                   (-1.0 / 2.0, dT[0])),
  T[2])

system.add_solve(inner(test, dT[n + 2]) * dx ==
  dt * F(test, T[n + 2]),
  dT[n + 2], solver_parameters = solver_parameters)
system.add_solve(LinearCombination((1.0, T[n + 2]),
                                   (23.0 / 12.0, dT[n + 2]),
                                   (-4.0 / 3.0, dT[n + 1]),
                                   (5.0 / 12.0, dT[n])),
  T[n + 3])
\end{lstlisting}
At the end of the simulation the final model solution is stored in
\verb=T[N + 2]=.

\section{Leapfrog}

A centred discretisation in time leads to:
\begin{equation}
  T^{n + 1} = T^{n - 1} + 2 \Delta t F( T^n ).
\end{equation}
This method is also known as the leapfrog scheme, and has second order accuracy.
This scheme cannot be used to perform the first model timestep as $T^{-1}$ is
not available. A single forward Euler step \ref{sect:fe} can be applied:
\begin{equation}
  T^1 = T^0 + \Delta t F(T^0).
\end{equation}

This discretisation can be implemented via:
\begin{lstlisting}
levels    = TimeLevels(levels = [n, n + 1, n + 2],
  cycle_map = {n:n + 1, n + 1:n + 2}, last_past_level = n + 1)
levels_dT = TimeLevels(levels = [n, n + 1], cycle_map = {n:n + 1},
  last_past_level = n)
T  = TimeFunction(levels,    space, name = "T")
dT = TimeFunction(levels_dT, space, name = "dT")

system = TimeSystem()

system.add_solve(T_ic, T[0])

system.add_solve(inner(test, dT[0]) * dx == dt * F(test, T[0]),
  dT[0], solver_parameters = solver_parameters)
system.add_solve(LinearCombination((1.0, T[0]),
                                   (1.0, dT[0])),
  T[1])

system.add_solve(inner(test, dT[n + 1]) * dx ==
  dt * F(test, T[n + 1]),
  dT[n + 1], solver_parameters = solver_parameters)
system.add_solve(LinearCombination((1.0, T[n]),
                                   (2.0, dT[n + 1])),
  T[n + 2])
\end{lstlisting}
At the end of the simulation the final model solution is stored in
\verb=T[N + 1]=.

\section{Adams-Moulton}

Adams-Moulton schemes take the form:
\begin{equation}\label{eqn:adams_moulton}
  T^{n + 1} = T^n + \Delta t \sum_{i = 0}^N \gamma_i F( T^{n + 1 - i}),
\end{equation}
where the $\gamma_i$ are constants. For $n > 1$ Adams-Moulton schemes therefore
require the values of $F(T)$ from $N - 1$ previous time levels. Given a choice
of $N$, the $\gamma_i$ are chosen so as to yield $N$th order accuracy. If
$N > 2$ then the scheme is not self starting, and care is required when
initialising in order to maintain the accuracy of the method. Further details of
Adams-Moulton schemes can be found in \citet[section III.1]{hairer1987}.

In an Adams-Moulton scheme the right-hand-side of equation
\eqref{eqn:adams_moulton} depends implicitly on the future solution $T^{n + 1}$.
If $F(T)$ is linear then the implicit term contributes to the left-hand-side
matrix in a linear solve for $T^{n + 1}$. If $F(T)$ is non-linear then
a non-linear equation for $T^{n + 1}$ must be solved.

\subsection{Backward Euler}\label{sect:be}

For $N = 0$ the single coefficient $\gamma_0$ is chosen to be:
\begin{equation}
  \gamma_0 = 1,
\end{equation}
yielding:
\begin{equation}
  T^{n + 1} = T^n + \Delta t F(T^{n + 1}).
\end{equation}
This is the backward Euler scheme, and has first order accuracy.

This discretisation can be implemented via:
\begin{lstlisting}
levels = TimeLevels(levels = [n, n + 1], cycle_map = {n:n + 1})
T = TimeFunction(levels, space, name = "T")

system = TimeSystem()

system.add_solve(T_ic, T[0])

system.add_solve(inner(test, T[n + 1]) * dx ==
  inner(test, T[n]) * dx + dt * F(test, T[n + 1]),
  T[n + 1], solver_parameters = solver_parameters)
\end{lstlisting}
Note that \verb+solver_parameters+ should define non-linear solver parameters if
$F(T)$ is non-linear.

\subsection{Implicit trapezium rule}\label{sect:implicit_trapezium}

For $N = 1$ the $\gamma_i$ are chosen to be:
\begin{align}
  \gamma_0 = \frac{1}{2}, & & \gamma_1 = \frac{1}{2},
\end{align}
yielding:
\begin{equation}
  T^{n + 1} = T^n + \Delta t \left[ \frac{1}{2} F(T^{n + 1}) + \frac{1}{2} F(T^n) \right].
\end{equation}
This is the implicit trapezium rule, and has second order accuracy. If $F(T)$ is
linear then this scheme is identical to the implicit midpoint rule
\ref{sect:implicit_midpoint}. This method is also known as the Crank-Nicolson
scheme \citep[e.g.][section 3.4]{donea2003}, (although, confusingly,
``Crank-Nicolson'' can refer to the implicit midpoint rule).

This discretisation can be implemented via:
\begin{lstlisting}
levels = TimeLevels(levels = [n, n + 1], cycle_map = {n:n + 1})
T = TimeFunction(levels, space, name = "T")

system = TimeSystem()

system.add_solve(T_ic, T[0])

system.add_solve(inner(test, T[n + 1]) * dx ==
  inner(test, T[n]) * dx +
  dt * (0.5 * F(test, T[n + 1]) + 0.5 * F(test, T[n])),
  T[n + 1], solver_parameters = solver_parameters)
\end{lstlisting}
Note that \verb+solver_parameters+ should define non-linear solver parameters if
$F(T)$ is non-linear.

\section{Implicit midpoint rule}\label{sect:implicit_midpoint}.

The implicit midpoint rule takes the form:
\begin{equation}
  T^{n + 1} = T^n + \Delta t F \left( \frac{1}{2} T^{n + 1} + \frac{1}{2} T^n \right).
\end{equation}
This has second order accuracy. If $F(T)$ is linear then this scheme is
identical to the implicit trapezium rule \ref{sect:implicit_trapezium}. This
method is also known as the Crank-Nicolson scheme
\citep[e.g.][section 2.22]{mitchell1980} (although, confusingly,
``Crank-Nicolson'' can refer to the implicit trapezium rule).

This discretisation can be implemented via:
\begin{lstlisting}
levels = TimeLevels(levels = [n, n + 1], cycle_map = {n:n + 1})
T = TimeFunction(levels, space, name = "T")

system = TimeSystem()

system.add_solve(T_ic, T[0])

system.add_solve(inner(test, T[n + 1]) * dx ==
  inner(test, T[n]) * dx +
  dt * F(test, 0.5 * T[n + 1] + 0.5 * T[n]),
  T[n + 1], solver_parameters = solver_parameters)
\end{lstlisting}
Note that \verb+solver_parameters+ should define non-linear solver parameters if
$F(T)$ is non-linear.

\section{Explicit Runge-Kutta}

Runge-Kutta schemes are a class of multi-stage integration schemes. In the
context of equation \eqref{eqn:t_timestep} an $S$-stage explicit Runge-Kutta
scheme takes the form:
\begin{subequations}
  \begin{equation}
    F^{n + 1,i} = F \left( T^n + \Delta t \sum_{j = 1}^{i - 1} \gamma_{ij} F^{n + 1,j} \right) \quad \textrm{ for } i \in \left\{ 1, \ldots, S \right\},
  \end{equation}
  \begin{equation}
    T^{n + 1} = T^n + \Delta t \sum_{j = 1}^S b_j F^{n + 1,j}.
  \end{equation}
\end{subequations}
The $\gamma_{ij}$ and $b_j$ are chosen so as to yield desired accuracy and
stability properties. Further details of explicit Runge-Kutta schemes can be
found in \citet[section 3.2]{iserles2009}.

\subsection{First order explicit Runge-Kutta}

The one stage first order explicit Runge-Kutta scheme is:
\begin{subequations}
  \begin{equation}
    F^{n + 1,1} = F \left( T^n \right),
  \end{equation}
  \begin{equation}
    T^{n + 1} = T^n + \Delta t F^{n + 1,1}.
  \end{equation}
\end{subequations}
This is the forward Euler scheme \ref{sect:fe}.

\subsection{Second order explicit Runge-Kutta}\label{sect:rk2}

A simple second order explicit Runge-Kutta scheme takes the form:
\begin{subequations}
  \begin{equation}
    F^{n + 1,1} = F \left( T^n \right),
  \end{equation}
  \begin{equation}
    F^{n + 1,2} = F \left( T^n + \Delta t \frac{1}{2} F^{n + 1,1} \right),
  \end{equation}
  \begin{equation}
    T^{n + 1} = T^n + \Delta t F^{n + 1,2}.
  \end{equation}
\end{subequations}

This discretisation can be implemented via:
\begin{lstlisting}
from fractions import Fraction
hf = Fraction(1, 2)

levels   = TimeLevels(levels = [n, n + 1], cycle_map = {n:n + 1})
levels_F = TimeLevels(levels = [n, n + hf], cycle_map = {},
  last_past_level = n - hf)
T    = TimeFunction(levels,   space, name = "T")
F_s  = TimeFunction(levels_F, space, name = "F_s")

system = TimeSystem()

system.add_solve(T_ic, T[0])

system.add_solve(inner(test, F_s[n]) * dx == F(test, T[n]),
  F_s[n], solver_parameters = solver_parameters)
system.add_solve(inner(test, F_s[n + hf]) * dx ==
  F(test, T[n] + 0.5 * dt * F_s[n]),
  F_s[n + hf], solver_parameters = solver_parameters)
  
system.add_solve(LinearCombination((1.0, T[n]),
                                   (dt, F_s[n + hf])),
  T[n + 1])
\end{lstlisting}

\section{Discontinuous Galerkin Finite Element}

Let a time interval $(0, T)$ be divided into a series of $N$ elements
$( (n - 1) \Delta t, n \Delta t )$,
$n \in \left\{ 1, \ldots, N \right\}$. Equip this one dimensional mesh with
discontinuous Lagrange basis functions, thus defining a discrete function
space $V^\delta \in L^2$ on $(0, T)$. A temporal discontinuous Galerkin
discretisation of equation \eqref{eqn:t_timestep} yields:
\begin{align}\label{eqn:t_timestep_dg}
  \left[ \phi^\delta T^{\delta,-} \right]_{(n - 1) \Delta t}^{n \Delta t}
    - \int_{(n - 1) \Delta t}^{n \Delta t} d_t \phi^\delta T^\delta dt
    = \int_{(n - 1) \Delta t}^{n \Delta t} \phi^\delta F \left( T^\delta \right) dt \nonumber \\
    \forall \phi^\delta \in V^\delta, \quad \forall n \in \left\{ 1, \ldots, N \right\},
\end{align}
where $T^\delta \in V^\delta$ and
$T^{\delta,-} = \lim_{t \rightarrow \left[ (n - 1) \Delta t \right]^{-}} T^{\delta}$.
The time derivative has been integrated by parts and an ``upwind flux''
approximation applied. For further details of discontinuous Galerkin in time
methods see \citet{gresho2000}.

\subsection{$P0$}

Consider discontinuous Lagrange elements of degree zero. Introduce basis
functions:
\begin{equation}
  \phi^n = \left\{ \begin{array}{l} 1 \quad \textrm{ if } t \in \left( (n - 1) \Delta t, n \Delta t \right) \\
                                    0 \quad \textrm{ otherwise}\end{array} \right. ,
\end{equation}
and let $T^\delta = \sum_{n = 1}^N T^n \phi^n$. Then equation
\eqref{eqn:t_timestep_dg} leads to:
\begin{equation}
  T^{n + 1} - T^n
    = \int_{n \Delta t}^{(n + 1) \Delta t} \phi^{n + 1} F \left( T^\delta \right) dt \quad \forall n \in \left\{ 0, \ldots, N - 1 \right\}.
\end{equation}
Application of the midpoint rule to the right-hand-side yields:
\begin{equation}
  T^{n + 1} - T^n = \Delta t F \left( T^{n + 1} \right).
\end{equation}
This is the backward Euler scheme \ref{sect:be}, and has first order accuracy
in time.

\subsection{$P1_{DG}$}

Consider discontinuous Lagrange elements of degree one. Introduce basis
functions:
\begin{subequations}
  \begin{align}
    \phi^{n,1} & = \left\{ \begin{array}{l} 1 - \frac{n \Delta t - t}{\Delta t} \\ 0 \end{array}\begin{array}{l} \textrm{ if } t \in \left( (n - 1) \Delta t, n \Delta t \right) \\ \textrm{ otherwise} \end{array} \right. , \\
    \phi^{n,2} & = \left\{ \begin{array}{l} \frac{t - (n - 1) \Delta t}{\Delta t} \\ 0 \end{array}\begin{array}{l} \textrm{ if } t \in \left( (n - 1) \Delta t, n \Delta t \right) \\ \textrm{ otherwise} \end{array} \right. ,
  \end{align}
\end{subequations}
and let $T^\delta = \sum_{n = 1}^N T^{n,1} \phi^{n,1} + T^{n,2} \phi^{n,2}$.
Then equation \eqref{eqn:t_timestep_dg} leads to:
\begin{subequations}
  \begin{align}
    -T^{n,2} + \frac{1}{2} \left( T^{n + 1,1} + T^{n + 1,2} \right)
      = \int_{n \Delta t}^{(n + 1) \Delta t} \phi^{n + 1,1} F \left( T^\delta \right) dt \nonumber \\ \forall n \in \left\{ 0, \ldots, N - 1 \right\},
  \end{align}
  \begin{align}
    T^{n + 1,2} - \frac{1}{2} \left( T^{n + 1,1} + T^{n + 1,2} \right)
      = \int_{n \Delta t}^{(n + 1) \Delta t} \phi^{n + 1,2} F \left( T^\delta \right) dt \nonumber \\ \forall n \in \left\{ 0, \ldots, N - 1 \right\},
  \end{align}
\end{subequations}
The right-hand-sides can be integrated to second order accuracy in time via
application of the trapezium rule to yield:
\begin{subequations}
  \begin{align}
    -T^{n,2} + \frac{1}{2} \left( T^{n + 1,1} + T^{n + 1,2} \right)
      = \Delta t \left[ \frac{1}{3} F \left( T^{n + 1,1} \right) + \frac{1}{6} F \left( T^{n + 1,2} \right) \right] \nonumber \\ \forall n \in \left\{ 0, \ldots, N - 1 \right\},
  \end{align}
  \begin{align}
    T^{n + 1,2} - \frac{1}{2} \left( T^{n + 1,1} + T^{n + 1,2} \right)
      = \Delta t \left[ \frac{1}{6} F \left( T^{n + 1,1} \right) + \frac{1}{3} F \left( T^{n + 1,2} \right) \right]  \nonumber \\ \forall n \in \left\{ 0, \ldots, N - 1 \right\},
  \end{align}
\end{subequations}
The complete discretisation, including the initial condition, becomes:
\begin{subequations}
  \begin{align}
    \frac{1}{2} T^{n + 1,2} + \frac{1}{2} T^{n + 1,1} - T^{n,2}
      = \Delta t \left[ \frac{1}{3} F \left( T^{n + 1,1} \right) + \frac{1}{6} F \left( T^{n + 1,2} \right) \right] \nonumber \\ \forall n \in \left\{ 0, \ldots, N - 1 \right\},
  \end{align}
  \begin{align}
    \frac{1}{2} T^{n + 1,2} - \frac{1}{2} T^{n + 1,1}
      = \Delta t \left[ \frac{1}{6} F \left( T^{n + 1,1} \right) + \frac{1}{3} F \left( T^{n + 1,2} \right) \right]  \nonumber \\ \forall n \in \left\{ 0, \ldots, N - 1 \right\},
  \end{align}
  \begin{equation}
    T^{0,2} = T_0.
  \end{equation}
\end{subequations}
Each timestep involves the implicit solution for two new values, $T^{n + 1,1}$
and $T^{n + 1,2}$, given one previous value, $T^{n,2}$.

This discretisation can be implemented via:
\begin{lstlisting}
spaces = FunctionSpace(mesh, space.ufl_element() * space.ufl_element())
tests = TestFunction(spaces)
test1, test2 = tests

levels = TimeLevels(levels = [n, n + 1], cycle_map = {n:n + 1})
T = TimeFunction(levels, spaces, name = "T")

system = TimeSystem()

system.add_solve(inner(tests, T[0]) * dx == inner(test2, T_ic) * dx,
  T[0], solver_parameters = solver_parameters)

m_11 = m_22 = 1.0 / 3.0
m_12 = m_21 = 1.0 / 6.0
system.add_solve(
  inner(test1, 0.5 * T[n + 1][1] + 0.5 * T[n + 1][0] - T[n][1])*dx +
  inner(test2, 0.5 * T[n + 1][1] - 0.5 * T[n + 1][0]) * dx ==
  dt*(m_11 * F(test1, T[n + 1][0]) + m_12 * F(test1, T[n + 1][1]) +
      m_21 * F(test2, T[n + 1][0]) + m_22 * F(test2, T[n + 1][1])),
  T[n + 1], solver_parameters = solver_parameters)
\end{lstlisting}
Note that \verb+solver_parameters+ should define non-linear solver parameters if
$F(T)$ is non-linear.

\bibliography{bibliography}
\bibliographystyle{plainnat}

\end{document}
